\part{}
\makeoddhead{modruled}{}{\scshape The Death of the Gods}{}
\markboth{The Death of the Gods}{}

\chapter{} %c1

Twenty stadia from Caesarea in Cappadocia, on the wooded peaks of Mount
Arg\ae us, near the great Roman road, was a source of warm, healing
water. A stone slab with crudely carved human figures and a Greek
inscription attested that this spring was once consecrated to the
Dioscuri---the brothers Castor and Pollux. These representations of
pagan gods, which remained untouched, were considered to be
representations of the Christian saints Cosmas and Damian.

On the other side of the road, across from the Sacred Spring, a small
\textit{taberna} had been built, a hut with a thatched roof, with a
dirty stockyard and a rough shelter for chickens and geese. Inside the
pub, goat cheese, country bread, honey, olive oil, and a fairly sour
local wine were on offer. A crafty Armenian named Syrax owned the
\textit{taberna}.

A partition divided it in two: one side was for simple folks, the
other for more esteemed guests. In the rafters, which were blackened
by acrid smoke, hung cured hams and bundles of fragrant mountain
grasses: Syrax's wife, Fortunata, was a good hostess.

The house was considered disreputable. Good people would not stay
the night there; there were rumors of dark deeds being done in that
hut. But Syrax was slippery, knew how to put bribes in the right
hands, and always came out smelling like a rose.

The partition consisted of two thin columns on which an old, faded
\textit{chlamys} of Fortunata's was stretched in place of a
curtain. The columns represented the pub's only luxury, and Syrax's
pride and joy: once gilded, they had long since started to crack and
peel; the dusty light blue fabric of the \textit{chlamys}, previously
bright lilac, was covered with patches and the stains of breakfasts,
lunches, and dinners, and reminded the virtuous Fortunata of ten years
of family life.

On the clean side, behind the curtain, Marcus Scudilo, the Roman
military tribune of the Sixteenth Legion, Ninth Cohort, reclined on
the only couch, which was narrow and worn, next to a table with a tin
\textit{krater} and bowls of wine. Marcus was the provincial dandy,
with the kind of face that made easy slave women and cheap
\textit{het\ae r\ae} on the outskirts of the city shout with simple
pleasure, ``What a beautiful man!'' At his feet, on the same
\textit{lectica}, a fat, red-faced man sat in a deferential and
uncomfortable posture, suffering from shortness of breath, his head
bald with sparse, grey hair combed forward toward the
temples---Publius Aquilus, centurion of the Eighth Centuria. Some
distance away, on the floor, twenty Roman legionaries were playing dice.

``I swear to Hercules,'' Scudilo shouted, ``I would rather be last in
Constantinople than first in this hole. Is this living, Publius? Come,
answer me with a clear conscience---is this living? To know that there
is nothing ahead of us but exercises and barracks and camps. You could
disappear in this stinking swamp and never see the light of day
again!'' %scudilo

``Yes, life here is unhappy, you could say,'' agreed Publius. ``It is
quiet, though.'' %publius

