\part{}
\makeoddhead{modruled}{}{\scshape The Death of the Gods}{}
\markboth{The Death of the Gods}{}

\chapter{} %c1

Twenty stadia from C\ae sarea in Cappadocia, on the wooded peaks of Mount Arg\ae us, near the great Roman road, there was a warm healing spring. A stone slab with crudely carved human figures and a Greek inscription attested that this spring was once consecrated to the Dioscuri---the brothers Castor and Pollux. These representations of pagan gods, which remained untouched, were considered to be representations of the Christian saints Cosmas and Damian.

On the other side of the road, across from the sacred spring, a small \textit{taberna} had been built, a hut with a thatched roof, a dirty stockyard, and a rough shelter for chickens and geese. Inside the pub, goat cheese, country bread, honey, olive oil, and a fairly sour local wine were on offer. A crafty Armenian named Syrax owned the \textit{taberna}.

A partition divided it in two: one side was for simple folks, the other for more esteemed guests. From the rafters, which were blackened by acrid smoke, hung cured hams and bundles of fragrant mountain grasses: Syrax's wife, Fortunata, was a good hostess.

The house was considered disreputable. People of good repute would not stay the night; there were rumors of dark things being done in that hut. But Syrax was slippery, knew how to put bribes in the right hands, and always came out smelling like a rose.

The partition consisted of two thin columns on which an old, faded \textit{chlamys} of Fortunata's was stretched in place of a curtain. The columns represented the pub's only luxury, and Syrax's pride and joy: once gilded, they had long since started to crack and peel; the dusty light blue fabric of the \textit{chlamys}, previously bright lilac, was covered with patches and the stains of many breakfasts, lunches, and dinners. They reminded the virtuous Fortunata of ten years of family life.

On the cleaner side, behind the curtain, Marcus Scudilo, the Roman military tribune of the Sixteenth Legion, Ninth Cohort, reclined on the only couch, which was narrow and worn, next to a table with a tin \textit{krater} and bowls of wine. Marcus was the provincial dandy, with the kind of face that made easy slave women and cheap \textit{het\ae r\ae} on the outskirts of the city shout with simple pleasure, ``What a gorgeous man!'' At his feet, on the same \textit{lectica}, a fat, red-faced man sat in a deferential and uncomfortable posture, suffering from shortness of breath, his head
bald with sparse, grey hair combed forward toward the temples---Publius Aquilus, centurion of the Eighth Centuria. Some distance away, on the floor, twenty Roman legionaries were playing dice.

``I swear to Hercules,'' Scudilo shouted, ``I would rather be last in Constantinople than first in this hole. You call this living, Publius? Come on, answer me with a clear conscience---you call this living? To know that there is nothing ahead of us but exercises and barracks and camps. You could
disappear in this stinking swamp and never see the light of day again!'' %scudilo

``Yes, life here is unhappy, you could say that,'' agreed Publius. ``It is quiet, though.'' %publius

The old centurion was paying attention to the dice; he pretended to listen to his commander's chatter, nodding at him, surreptiously following the soldiers' game and thinking, \textit{If that redheaded soldier can manage a good roll, he'll probably win.} For propriety's sake, trying to seem as if he really cared, Publius asked the tribune:

``Now why exactly did you say that prefect Helvidius is angry at you?'' %publius

``Because of a woman, my friend, it's always because of a woman.'' %scudilo

In an attack of openness, his face taking on a secretive look, Marcus chattered in the centurion's ear, saying that the prefect, ``that old goat Helvidius,'' was jealous of the attention he was getting from a visiting Lilyb\ae an \textit{het\ae ra}; Scudilo wanted to return to Helvidius' good graces by doing some kind of valuable service for him. Not far from C\ae sarea, in the fortress of Macellum, Julian and Gallus, the final, unfortunate offspring of the House of the Flavii, were locked up, cousins of the reigning emperor, Constantius, and nephews of Constantine the Great. Upon ascending the throne, Constantius, fearing his rivals, did away with his own uncle, Julius Constantius, Julian and Gallus' father, Constantine's brother. Many others fell victim to his fear. But Julius and Gallus were spared, exiled to the secluded castle of Macellum. The Prefect of C\ae sarea, Helvidius, was in a very difficult position. Knowing that the new emperor hated the two boys, who reminded him of his crime, Helvidius wanted to divine Constantius' will toward them, and was also afraid to do so. Julian and Gallus lived under constant fear of death. The cunning tribune Scudilo, dreaming of the possibility of being rewarded at court, understood from his commander's hints that he could not decide to take the responsibility for dealing with them on himself and was scared by tales he had heard that Constantine's heirs were planning an escape; Marcus decided then to set out for Macellum with a detachment of legionaries and seize the prisoners on his own authority so that he could bring them out to C\ae sarea, thinking that he had nothing to fear from two underaged boys who had been abandoned by everyone, orphans whom the emperor hated. He hoped that by this deed he could return to the prefect Helvidius' good graces, having fallen out of favor due to the redheaded Lilyb\ae an girl.

However, Marcus communicated only part of his intentions to Publius, and carefully at that.

``What are you planning on doing, Scudilo? You haven't gotten any orders from Constantinople, have you?'' %publius

``No orders of any kind; no one knows anything for sure. But there are rumors, you see? Thousands of different rumors and expectations, and hints, and innuendos, and threats, and secrets---there's no end to the secrets! Any idiot can do what he's told. But if you can guess the unspoken will of the powerful---they'll thank you for that. We'll search, try out a few things, look around. The most important thing is to be bold, to act boldy having taken on the sign of the cross. I am counting on you, Publius. We may be drinking wine sweeter than this at court soon\ldots{}'' %scudilo

The dismal light of an overcast evening fell through the small, latticed window; the sound of the rain was monotonous.

Nearby, on the other side of a thin clay wall with many cracks, was a pigsty; it stunk of manure, and one could hear chickens clucking, chicks peeping, and pigs grunting; milk was spraying loudly into a jug: the hostess was apparently milking a cow.

The soldiers, fighting over the pot, were swearing in a loud whisper. Right near the floor, the tender, rosy face of a piglet was poking through a crack between willow branches that were not sufficiently covered with clay; he had become trapped, could not pull out his head, and was screeching piteously.

Publius thought, \textit{Well, at the moment we're closer to the stockyard than any courtyard.}

His anxiety passed. The tribune also started to get bored after his excessive chattering. He looked at the rainy grey sky through the window, at the clay-stained face of the piglet, at the sour dregs of the bad wine in the tin bowl, at the dirty soldiers, and he was overcome with rage.
