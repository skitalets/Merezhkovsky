\part{}
\makeoddhead{modruled}{}{\scshape The Death of the Gods}{}
\markboth{The Death of the Gods}{}

\chapter{} %c1

Twenty stadia from Caesarea in Cappadocia, on the wooded peaks of Mount
Arg\ae us near the main Roman road, there was a source of warm,
healing water. A stone slab with crudely carved human figures and a
Greek inscription attested that this spring was once consecrated to
the Dioscuri---the brothers Castor and Pollux. These representations
of pagan gods, which remained untouched, were now considered
representations of the Christian saints Cosmas and Damian.

On the other side of the road, facing the sacred Spring, a small
\textit{taberna} had been built, a thatched hut with a dirty cattle
yard and a canopy for chickens and geese. In the pub, one could get
goat's cheese, half-white bread, honey, olive oil, and a rough, honest
wine. The \textit{taberna} was owned by a shrewd Armenian, Syrax.

A partition divided it into two parts: one for the common people, and the
other for more honorable guests. Just below the ceiling, which had
turned black from the acrid smoke, hung cured hams and bunches of
fragrant mountain herbs: Syrax's wife, Fortunata, was a good hostess.

The house was considered disreputable. Good people would not stay
the night there; there were tales of dark business being transacted in
that hut. But Syrax was slippery, knew how to put bribes in the right
hands, and always came out of water dry. (FIX)

The partition consisted of two thin columns with an old, faded
\textit{chlamys} of Fortunata's stretched between them in place of a
curtain.\footnote{A \textit{chlamys} was a short cloak worn in ancient Greece
  and the Byzantine Empire. --- Trans.} These columns were the pub's
only luxury and were the pride of Syrax: once gilded, they had long
since cracked and begun to peel; the fabric of the \textit{chlamys},
once a bright violet, was now a dusty light blue, and was spotted with
numerous patches and the traces of breakfasts, lunches, and dinners
that reminded the virtuous Fortunata of ten years of family life.

In the clean half, separated by the curtain, the Roman military
tribune Marcus Scudilo of the Sixteenth Legion, Ninth Cohort, lay on
the single narrow, worn couch before a table with a tin
\textit{krater} and bowls of wine.\footnote{A \textit{krater} was a
  large vessel used for mixing wine with water. --- Trans.} Marcus was
the provincial dandy, with the kind of face that made easy slave women
and cheap \textit{het\ae r\ae} in the city's suburbs cry out in pure
delight: ``What a beautiful man!''\footnote{\textit{Het\ae r\ae} were
  courtesans, paid female companions. --- Trans.} At his feet, on the
same \textit{lectica}, sat a fat, red-faced man who was suffering from
shortness of breath, his body in a deferential and uncomfortable
position, his skull bare with sparse grey hair combed back at his
temples---Publius Aquila, centurion of the Eighth Centuria. Some
distance away, on the floor, twelve Roman legionaries were playing
dice.

``I swear by Hercules,'' Scudilo shouted, ``it would be better for me
to be last in Constantinople than to be first in this hole! Is this
really living, Publius? Well, answer me with a clear conscience, is
this really living? When you know that there's nothing ahead but
exercises and barracks and camps. You can vanish in a stinking swamp
and never see light again!'' %scudilo

``Yes, life here is\ldots{}you could say it's unhappy,'' Publius
agreed. ``But then, it's peaceful.'' %publius

